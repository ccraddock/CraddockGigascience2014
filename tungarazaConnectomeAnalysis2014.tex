%% BioMed_Central_Tex_Template_v1.06
%%                                      %
%  bmc_article.tex            ver: 1.06 %
%                                       %

%%IMPORTANT: do not delete the first line of this template
%%It must be present to enable the BMC Submission system to
%%recognise this template!!

%%%%%%%%%%%%%%%%%%%%%%%%%%%%%%%%%%%%%%%%%
%%                                     %%
%%  LaTeX template for BioMed Central  %%
%%     journal article submissions     %%
%%                                     %%
%%          <8 June 2012>              %%
%%                                     %%
%%                                     %%
%%%%%%%%%%%%%%%%%%%%%%%%%%%%%%%%%%%%%%%%%


%%%%%%%%%%%%%%%%%%%%%%%%%%%%%%%%%%%%%%%%%%%%%%%%%%%%%%%%%%%%%%%%%%%%%
%%                                                                 %%
%% For instructions on how to fill out this Tex template           %%
%% document please refer to Readme.html and the instructions for   %%
%% authors page on the biomed central website                      %%
%% http://www.biomedcentral.com/info/authors/                      %%
%%                                                                 %%
%% Please do not use \input{...} to include other tex files.       %%
%% Submit your LaTeX manuscript as one .tex document.              %%
%%                                                                 %%
%% All additional figures and files should be attached             %%
%% separately and not embedded in the \TeX\ document itself.       %%
%%                                                                 %%
%% BioMed Central currently use the MikTex distribution of         %%
%% TeX for Windows) of TeX and LaTeX.  This is available from      %%
%% http://www.miktex.org                                           %%
%%                                                                 %%
%%%%%%%%%%%%%%%%%%%%%%%%%%%%%%%%%%%%%%%%%%%%%%%%%%%%%%%%%%%%%%%%%%%%%

%%% additional documentclass options:
%  [doublespacing]
%  [linenumbers]   - put the line numbers on margins

%%% loading packages, author definitions

\documentclass[twocolumn]{bmcart}% uncomment this for twocolumn layout and comment line below
%\documentclass{bmcart}

%%% Load packages
\usepackage{amsthm,amsmath}
\usepackage{siunitx}
%\RequirePackage{natbib}
\usepackage[colorinlistoftodos]{todonotes}
\RequirePackage{hyperref}
\usepackage[utf8]{inputenc} %unicode support
%\usepackage[applemac]{inputenc} %applemac support if unicode package fails
%\usepackage[latin1]{inputenc} %UNIX support if unicode package fails

\usepackage{array}
\newcolumntype{L}[1]{>{\raggedright\let\newline\\\arraybackslash\hspace{0pt}}p{#1}}

%%%%%%%%%%%%%%%%%%%%%%%%%%%%%%%%%%%%%%%%%%%%%%%%%
%%                                             %%
%%  If you wish to display your graphics for   %%
%%  your own use using includegraphic or       %%
%%  includegraphics, then comment out the      %%
%%  following two lines of code.               %%
%%  NB: These line *must* be included when     %%
%%  submitting to BMC.                         %%
%%  All figure files must be submitted as      %%
%%  separate graphics through the BMC          %%
%%  submission process, not included in the    %%
%%  submitted article.                         %%
%%                                             %%
%%%%%%%%%%%%%%%%%%%%%%%%%%%%%%%%%%%%%%%%%%%%%%%%%


\def\includegraphic{}
\def\includegraphics{}

%%% Put your definitions there:
\startlocaldefs
\endlocaldefs


%%% Begin ...
\begin{document}

%%% Start of article front matter
\begin{frontmatter}

\begin{fmbox}
\dochead{Review}

%%%%%%%%%%%%%%%%%%%%%%%%%%%%%%%%%%%%%%%%%%%%%%
%%                                          %%
%% Enter the title of your article here     %%
%%                                          %%
%%%%%%%%%%%%%%%%%%%%%%%%%%%%%%%%%%%%%%%%%%%%%%

\title{Connectomics and open science approaches to analyze brain connectivity}

%%%%%%%%%%%%%%%%%%%%%%%%%%%%%%%%%%%%%%%%%%%%%%
%%                                          %%
%% Enter the authors here                   %%
%%                                          %%
%% Specify information, if available,       %%
%% in the form:                             %%
%%   <key>={<id1>,<id2>}                    %%
%%   <key>=                                 %%
%% Comment or delete the keys which are     %%
%% not used. Repeat \author command as much %%
%% as required.                             %%
%%                                          %%
%%%%%%%%%%%%%%%%%%%%%%%%%%%%%%%%%%%%%%%%%%%%%%
%%
\author[
   addressref={aff1,aff2},
   corref={aff1},
   email={ccraddock@nki.rfmh.org}
]{\inits{RCC} \fnm{R. Cameron} \snm{Craddock}}
\author[
   addressref={aff1},                   % id's of addresses, e.g. {aff1,aff2}
   email={rosalia.tungaraza@childmind.org}   % email address
]{\inits{RLT} \fnm{Rosalia L.} \snm{Tungaraza}}
\author[
   addressref={aff1,aff2}, 
   email={michael.milham@childmind.org}
]{\inits{MPM} \fnm{Michael P.} \snm{Milham}}


%%%%%%%%%%%%%%%%%%%%%%%%%%%%%%%%%%%%%%%%%%%%%%
%%                                          %%
%% Enter the authors' addresses here        %%
%%                                          %%
%% Repeat \address commands as much as      %%
%% required.                                %%
%%                                          %%
%%%%%%%%%%%%%%%%%%%%%%%%%%%%%%%%%%%%%%%%%%%%%%
\address[id=aff1]{%                           % unique id
  \orgname{Computational Neuroimaging Lab, Center for Biomedical Imaging and Neuromodulation, Nathan Kline Institute for Psychiatric Research}, % university, etc
  \street{140 Old Orangeburg Rd},                     %
  \postcode{10962}                                % post or zip code
  \city{Orangeburg},                              % city
  \state{New York},
  \cny{USA}                                    % country
}
\address[id=aff2]{%                           % unique id
  \orgname{Center for the Developing Brain, Child Mind Institute}, % university, etc
  \street{445 Park Ave},                     %
  \postcode{10022}                                % post or zip code
  \city{New York},                              % city
  \state{New York},
  \cny{USA}                                    % country
}


%%%%%%%%%%%%%%%%%%%%%%%%%%%%%%%%%%%%%%%%%%%%%%
%%                                          %%
%% Enter short notes here                   %%
%%                                          %%
%% Short notes will be after addresses      %%
%% on first page.                           %%
%%                                          %%
%%%%%%%%%%%%%%%%%%%%%%%%%%%%%%%%%%%%%%%%%%%%%%

\begin{artnotes}
\end{artnotes}

%\end{fmbox}% comment this for two column layout

%%%%%%%%%%%%%%%%%%%%%%%%%%%%%%%%%%%%%%%%%%%%%%
%%                                          %%
%% The Abstract begins here                 %%
%%                                          %%
%% Please refer to the Instructions for     %%
%% authors on http://www.biomedcentral.com  %%
%% and include the section headings         %%
%% accordingly for your article type.       %%
%%                                          %%
%%%%%%%%%%%%%%%%%%%%%%%%%%%%%%%%%%%%%%%%%%%%%%

\begin{abstractbox}

\begin{abstract} % abstract
	
	Estimating the functional interactions between brain regions and mapping those connections to inter-individual variability are central pursuits for understanding the human connectome. The number and complexity of funcitonal interactions withen the connectome and the large amounts of data required to study them position functional connectivity research as a ``big data'' problem. Maximimizing the degree to which knowledge about human brain function can be extracted from the connectome will require developing a new generation of neuroimaging analysis algorithms and tools. This review describes several outstanding problems in brain functional connectivity research with the goal of engaging researchers from a broad specturm of data sciences to help solve these problems. Additionally it provides information about open science resources consisting of raw and preprocessed data to help interested researchers get started. 

\end{abstract}



%%%%%%%%%%%%%%%%%%%%%%%%%%%%%%%%%%%%%%%%%%%%%%
%%                                          %%
%% The keywords begin here                  %%
%%                                          %%
%% Put each keyword in separate \kwd{}.     %%
%%                                          %%
%%%%%%%%%%%%%%%%%%%%%%%%%%%%%%%%%%%%%%%%%%%%%%

\begin{keyword}
\kwd{human connectome}
\kwd{functional MRI}
\kwd{brain graphs}
\kwd{open data}
\kwd{open science}
\end{keyword}

% MSC classifications codes, if any
%\begin{keyword}[class=AMS]
%\kwd[Primary ]{}
%\kwd{}
%\kwd[; secondary ]{}
%\end{keyword}

\end{abstractbox}
%
\end{fmbox}% uncomment this for twcolumn layout

\end{frontmatter}

%%%%%%%%%%%%%%%%%%%%%%%%%%%%%%%%%%%%%%%%%%%%%%
%%                                          %%
%% The Main Body begins here                %%
%%                                          %%
%% Please refer to the instructions for     %%
%% authors on:                              %%
%% http://www.biomedcentral.com/info/authors%%
%% and include the section headings         %%
%% accordingly for your article type.       %%
%%                                          %%
%% See the Results and Discussion section   %%
%% for details on how to create sub-sections%%
%%                                          %%
%% use \cite{...} to cite references        %%
%%  \cite{koon} and                         %%
%%  \cite{oreg,khar,zvai,xjon,schn,pond}    %%
%%  \nocite{smith,marg,hunn,advi,koha,mouse}%%
%%                                          %%
%%%%%%%%%%%%%%%%%%%%%%%%%%%%%%%%%%%%%%%%%%%%%%

%%%%%%%%%%%%%%%%%%%%%%%%% start of article main body
% <put your article body there>

\input{tungarazaConnectomeAnalysis2014_newtext_v3}

%%%%%%%%%%%%%%%%%%%%%%%%%%%%%%%%%%%%%%%%%%%%%%
%%                                          %%
%% Backmatter begins here                   %%
%%                                          %%
%%%%%%%%%%%%%%%%%%%%%%%%%%%%%%%%%%%%%%%%%%%%%%

\begin{backmatter}

\section*{Competing interests}
  The authors declare that they have no competing interests.

\section*{Author's contributions}
    Text for this section \ldots

\section*{Acknowledgements}
  Text for this section \ldots
%%%%%%%%%%%%%%%%%%%%%%%%%%%%%%%%%%%%%%%%%%%%%%%%%%%%%%%%%%%%%
%%                  The Bibliography                       %%
%%                                                         %%
%%  Bmc_mathpys.bst  will be used to                       %%
%%  create a .BBL file for submission.                     %%
%%  After submission of the .TEX file,                     %%
%%  you will be prompted to submit your .BBL file.         %%
%%                                                         %%
%%                                                         %%
%%  Note that the displayed Bibliography will not          %%
%%  necessarily be rendered by Latex exactly as specified  %%
%%  in the online Instructions for Authors.                %%
%%                                                         %%
%%%%%%%%%%%%%%%%%%%%%%%%%%%%%%%%%%%%%%%%%%%%%%%%%%%%%%%%%%%%%

% if your bibliography is in bibtex format, use those commands:
\bibliographystyle{bmc-mathphys} % Style BST file
\bibliography{tungarazaConnectomeAnalysis2014} % Bibliography file (usually '*.bib' )

\section*{Figures}
  \begin{figure}[h!]
  \caption{\csentence{Sample figure title.}
      A short description of the figure content
      should go here.}
      \end{figure}

\begin{figure}[h!]
  \caption{\csentence{Sample figure title.}
      Figure legend text.}
      \end{figure}
	  
\section*{Tables}
\renewcommand{\arraystretch}{1.5}

\begin{table}[h!]
\caption{Sample table title. This is where the description of the table should go.}
      \begin{tabular}{L{2.5in} L{.25in} L{.35in} L{3in} L{.25in}}
		  Resource & N & Ages & Details & Ref \\ 
        \hline
%		\multicolumn{2}{l}{\href{http://fcon_1000.projects.nitrc.org}{1000 Functional Connectomes (FCP)}} \\
%		\\
%		\multicolumn{2}{l}{\href{https://thedata.harvard.edu/dvn/dv/GSP}{Brain Genomics Superstruct Project (GSP)}} \\	
%		\multicolumn{2}{l}{\href{http://fcon_1000.projects.nitrc.org}{International Neuroimaging Datasharing Initiative (INDI)}} \\
		\href{http://fcon_1000.projects.nitrc.org/indi/adhd200}{ADHD-200} & 1088 & 7 - 21 & 490 with ADHD and 598 HC; rfMRI, sMRI, DX, severity, and IQ & \\
%		\\
		\href{http://fcon_1000.projects.nitrc.org/indi/abide}{Autism Brain Imaging Data Exchange (ABIDE)} & 1112 & 6 - 64 & 539 with Autism and 573 HC; rfMRI, sMRI, DX, severity, and IQ &  \\
%		\\
%		& \href{http://fcon_1000.projects.nitrc.org/indi/CoRR/html/}{Consortium for Reliability and Reproducibility} \\
%		& \href{http://fcon_1000.projects.nitrc.org/indi/enhanced/}{Enhanced Nathan Kline Institute - Rockland Sample} \\
%		\multicolumn{2}{l}{\href{http://www.humanconnectomeproject.org/}{MGH-USC Human Connectome Project (HCP)}} \\
%		\multicolumn{2}{l}{\href{http://ndar.nih.gov/}{National Database for Autisim Research (NDAR)}} \\	
%		\multicolumn{2}{l}{\href{https://openfmri.org/}{OpenFMRI}} \\
%		\multicolumn{2}{l}{\href{http://pingstudy.ucsd.edu/}{Pediatric Imaging, Neurocognition and Genetics (PING) Study}} \\
		\href{http://www.med.upenn.edu/bbl/projects/pnc/PhiladelphiaNeurodevelopmentalCohort.shtml}{Philadelphia Neurodevelopmental Cohort} & 1445 & 8 - 21 & development cohort, individuals with a variety of diagnoses; sMRI, dMRI, tfMRI, rfMRI, CBF, neuropsychiatric assessment, genotyping, and computerized neurocognitive testing & \\
%		\multicolumn{2}{l}{\href{http://preprocessed-connectomes-project.github.io/}{Preprocessed Connectomes Project (PCP)} Preprocessed data and common statistical derivatives for resting state fMRI, structural MRI, and diffusion MRI scans for data shared through INDI.} \\
%		\multicolumn{2}{l}{\href{http://www.humanconnectome.org/}{WU-Minn Human Connectome Project (HCP)}} \\
%        \hline
      \end{tabular}
\end{table}

\end{backmatter}
\end{document}
