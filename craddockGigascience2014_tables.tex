\renewcommand{\arraystretch}{1.5}

\begin{table}[h!]
\caption{List of resources for openly shared raw and processed neuroimaging data. $^{\star}$repositories that include some overlap.}
      \begin{tabular}{L{5.5in}}
        \hline
        \href{http://fcon_1000.projects.nitrc.org}{1000 Functional Connectomes (FCP)}$^{\star}$: Raw resting state functional MRI and structural MRI for more than 1200 healthy individuals from 33 different contributors. \\ 
		\href{https://thedata.harvard.edu/dvn/dv/GSP}{Brain Genomics Superstruct Project (GSP)}$^{\star}$: Raw resting state functional MRI, and structural MRI data, along with automated quality assessment and pre-computed brain morphometrics, along with cognitive, personality, and behavior data for 1570 college age healthy individuals (18 - 35 years old) acquired using 1 of 4 MRI scanners. 1139 of the particpants have a second resting-state fMRI scans acquired from the same scan and 69 have re-test scans.\\
		\href{http://fcon_1000.projects.nitrc.org}{International Neuroimaging Datasharing Initiative (INDI)}$^{\star}$: Raw resting state functional MRI, task-based functional MRI, structural MRI, and diffusion MRI data for 20 different projects, 9 of which are being shared prospectively, as they are collected, and before publication. Contains data from a variety of different clinical populations and other experimental designs. Notable examples are the \href{http://fcon_1000.projects.nitrc.org/indi/adhd200}{ADHD-200}, which contains 490 individuals with ADHD and 598 typically developing controls, the \href{http://fcon_1000.projects.nitrc.org/indi/abide}{Autism Brain Imaging Data Exchange} (ABIDE; 539 Autism and 573 healthy controls), the \href{http://fcon_1000.projects.nitrc.org/indi/CoRR/html/}{Consortium for Reliability and Reproducibility (CoRR)}, which contains test-retest datasets on over 1600 individuals, and the \href{http://fcon_1000.projects.nitrc.org/indi/enhanced/}{Enhanced Nathan Kline Institute - Rockland Sample}, which is a community ascertained longitudinal sample with deep phenotyping. \\
        \href{http://www.humanconnectomeproject.org/}{Human Connectome Project (HCP)}: Resting state functional MRI, task functional MRI, structural MRI, diffusion MRI, deep phenotyping, and genetics collected on a variety of individuals including 1,200 healthy adults (twins and non-twin siblings) by two consortias, one between Washington University St. Louis and University of Minnesota and another between Massachutes General Hospital and University of Southern California.\\
        \href{http://ndar.nih.gov/}{National Database for Autisim Research (NDAR)}: An NIH-funded data repository of raw and preprocessed neuroimaging, phenotypic, and genomic data from a variety of different Autism experiments.\\	
        \href{https://openfmri.org/}{OpenFMRI}: Raw and preprocessed data along with behavioral data for a variety of different task-based functional MRI experiments. \\
        \href{http://pingstudy.ucsd.edu/}{Pediatric Imaging, Neurocognition and Genetics (PING) Study}: A multisite project that has collected "neurodevelopmental histories, information about mental and emotional functions, multimodal brain imaging data and genotypes for well over 1000 children and adolescents between the ages 3 and 20". \\
        \href{http://www.med.upenn.edu/bbl/projects/pnc/PhiladelphiaNeurodevelopmentalCohort.shtml}{Philadelphia Neurodevelopmental Cohort}: Raw structural MRI, diffusion MRI, task functional MRI, resting state fMRI, cerebral blood flow, neuropsychiatric assessment, genotyping, and computerized neurocognitive testing for 1445 individuals, 8 - 21 years old, including healthy controls and individuals with a variety of diagnoses.\\
        \href{http://preprocessed-connectomes-project.github.io/}{Preprocessed Connectomes Project (PCP)}: Preprocessed data and common statistical derivatives for resting state fMRI, structural MRI, and diffusion MRI scans for data shared through INDI. \\
        \hline
      \end{tabular}
\end{table}