\section{Introduction}

With its new emphasis on collecting larger datasets, data sharing, deep
phenotyping, and multimodal integration, neuroimaging has become a data
intensive science. This is particularly true for connectomics where grass-root
initiatives ([UPDATED a bit] e.g.the 1000 Functional Connectomes Project (FCP)
\cite{Biswal2010}, the International Neuroimaging Data-sharing Initiative
(INDI) \cite{Mennes2013}) and large-scale interational projects (the Human
Connectome Project\cite{Sotiropoulus2013,VanEssen2012}, the Brainnetome
\cite{Jiang2013}, the Human Brain Project in EU known as
CONNECT\cite{Assaf2013}, PING, Philadelphia Neurodevelopmental Cohort,  Brain Genomics Superstruct, NDAR \todo{i am not sure that all of these share
data [MPM: I believe all do/say they will, except the Brainnetome, which may at some point]}) 
are collecting and openly sharing thousands of brain imaging scans, each
of which consist of hundreds of observations of thousands of variables.
Although this deluge of complex data promises to enable the investigation of
neuroscientific questions that were previously inaccessible, it is quickly
overwhelming the capacity of existing tools and algorithms to extract
meaningful information from the data. This combined with a new focus on
discovery science is creating a plethora of opportunities for data scientists
from a wide range of disciplines such as computer science, engineering,
mathematics, statistics, etc., to make substantial contributions to
neuroscience. The goal of this review is to describe the state-of-the-art in
connectomics research and enumerate opportunities for data scientists to
contribute to the field.

The human connectome is a comprehensive map of the brain's circuitry, which
consists of brain areas, their structural connections and their functional
interactions. The connectome can be measured with a variety of different
imaging techniques, but magnetic resonance imaging (MRI) is the most common
primarily due to its near-ubiquity, non-invasiveness, and high spatial
resolution. As measured by MRI: brain areas are patches of cortex
(approximately 1\si{\centi\meter\cubed} volume) containing 1,000s of neurons,
structural connections are long range fiber tracts that are inferred from the
motion of water particles measured by diffusion weighted MRI (dMRI), and
functional interactions are inferred from synchronized brain activity measured
by functional MRI (fMRI). Addressing the current state-of-the-art for both
functional and structural connectivity is well beyond the scope of a single
review. Instead, this review will focus on functional connectivity, which is
particularly fast growing and offers many exciting opportunities for data
scientists.

The advent of functional connectivity analyses has popularized the application
of discovery science to brain function, which marks a shift in emphasis from
hypothesis testing, to supervised and unsupervised methods for learning
statistical relationships from the data. Since functional connectivity is
inferred from statistical dependencies between physiological measures of brain
activity (i.e. correlations between the dependent variables), it can be
measured without an experimental manipulation. Thus, functional connectivity is
most commonly measured from "resting state" fMRI scans, during which the study
participant is lying quietly and not performing an experimenter specified task
- when measured in this way, it is referred to as intrinsic functional
connectivity (iFC). Once iFC is measured, data mining techniques can be applied
to identify iFC patterns that covary with phenotypes, such as, indices of
cognitive abilities, personality traits, or disease state, severity, and
prognosis, to name a few. In a time dominated by skepticism about the
ecological validity of psychiatric diagnoses, iFC analyses have become
particularly important for identifying subgroups within patient populations by
similarity in brain architecture, rather than similarity in symptom profiles.
This new emphasis in discovery necessitates a new breed of data analysis tools
that are equipped to deal with the issues inherent to functional neuroimaging
data.

\section{The Connectome Analysis Paradigm}

In 2005 Sporn and Hagmann \cite{Sporns2005,Hagmann2005} independently and in
parallel coined the term \textit{the human connectome}, which embodies the
notion that the set of all connections within the human brain can be
represented and understood as graphs. In the context of iFC, graphs provide a
mathematical representation of the functional interactions between brain areas -
 nodes in the graph represent brain areas and edges indicate their functional connectivity.
While general graphs can have multiple edges between two nodes, brain
graphs tend to be simple graphs with a single undirected edge between pairs of
nodes (i.e. the direction of influence between nodes is unknown). Additionally
edges in graphs of brain function tend to be weighted - annotated with a value
that indicates the similarity between nodes. Analyzing the functional connectome  [CHANGE]
involves 1) preprocessing the data to remove confounding variation and to make
it comparable across datasets, 2) specification of brain areas to be used as nodes,
3) identification of edges from the iFC between nodes, and 4) analysis of the graph
(i.e. the structure and edges) to identifiy relationships with inter- or intra- individual
variability. All of these steps have been well covered in the literature \todo{cite many papers} by other
reviews and repeating that information provides little value. Instead we will focus on a 
exciting areas in the functional connectomics literature that we beleive provide the greatest
opportunities for data scientists in this quickly advancing field.

\subsection{Modeling connections}

\subsubsection{Static Connectivity}
A variety of different bivariate and multivariate methods have been proposed for measuring the
similarity between timecourses of brain areas \citep{SmithNeuor2010,Varoquaux}. Although these
methods are well suited for identifying weighted edges for connectome graphs, they provide an
incomplete picture of node interactions that limits the amount of neuroscientific information
that can be extracted.  Thus several different modeling techniques have been proposed 
to provide a more precise mathematical description of the relationship between brain areas. Model
confirmatory approaches such as structural equation modeling (SEM) and dynamic causal modeling (DCM)
offer fairly detailed descriptions of node relationships that have shown great promise in applications
such as predicting the affected hemisphere from stroke data \cite{GAJames}. But, they rely on the
prespecification of a model and are limited in the size of a network that can be modeled. Cross-validation
methods have been proposed to systematically search for the best model, but simulations have shown that
those methods do not necessarily converge to the correct model \cite{lohman}. Granger causality has 
been particularly popular due to its promise of identifying causal relationships between nodes based on
temporal lags between them \cite{}. But the assumptions underlying granger causality do not quite fit with 
fMRI data, where delays in the time-courses between regions may reflect some physiological 
phenomena, such as a perfusion deficit \cite{Lv}, rather than causal relationships between brain areas. 

Perhaps the oldest model of functional connectivity represents the activity of a single brain areas
or node as the weighted average of the activity measured in every other region of the brain \cite{Firston, 1993}. 
This multivariate regression model provides a more complete picture than commonly used bivariate measures, because the estimated 
coefficients describe a precise mathematical relationship, albeit not causal, between brain areas. Additionally this model
is primarily sensitive to direct, rather than indirect, interactions. Unfortunately due to the large number of brain areas in the connectome, and the 
few numbers of observations available standard resting state fMRI acquisitions, this model is underdetermined, and methods
that rely on either dimensionality reduction \cite{Friston1993} or regularization \cite{Gael, Craddock, etc} must be employed to find a unique solution. These methods have yet to become very popular for modeling connections, perhaps due to the complexity (real or perceived) in their use. One interesting
application of these multivariate regression approach, is that they can be applied to data from a different scanning session, experimental paradigm,
or even a different subjet to measure how well the model generalizes to the new data \cite{Craddock}. 

\subsection{Dynamic Connectivity} [?ALT: THE DYNAMIC CONNECTOME]
[ALT: Consider starting with the idea that: Although functional interactions within the connectome are commonly represented using three dimensional graphs, much like the structural wiring diagram for the connectome, recent work has drawn attention to time as a potential fourth dimension that requires consideration].

Standard seed- and ICA- methods for mapping iFC assume that it is stationary,
and derive connectivity patterns from the entirity of the available fMRI time
course. Recent studies however, have demonstrated that connectivity between
brain regions change dynamically over time \cite{Chang, Keilholz,
Hutchinson2013, Fu2013, Zhen}. A variety of investigations have dynamic iFC have
already been performed, most of which measure connectivity withen small a
window of the fMRI time course that is gradually moved forward along time
\cite{}. Several problems must be overcome in order to reliably measure
changing functional connectivity patterns, from the inherently slow and poorly
sampled fMRI signal. First, the variance of correlation estimates increases
with decreasing window size, meaning that unless proper statistical controls
are utilized, the observed dynamics may arise solely from the increased
variance \cite{}. This issue may be mitigated using the new higher speed
imaging methods, which has already shown promise for extracting dynamic network
modes using temporal ICA, although really large number of observations are
still necessary \cite{Smith2012}.\todo{verify this} Node definition is another
issue, as it is unclear whether brain areas defined from static iFC are
appropriate for dynamic iFC, although initial work has shown that parcellations
of at least some brain regions from dynamic iFC are consistent with what is
found with static \cite{Yang2013}.

\subsection{Comparing brain graphs} 

The ultimate goals of connectomics is to map the brain's functional
architecture and to annotate that architecture with the cognitive or behavioral
functions that they subtend. This latter pursuit is achieved by a group level
analysis in which variations in the connectome are mapped to inter-individual
variations in phenotype \cite{Kelly2011}, or intra-individual responses to
experimental perturbations \cite{Shirer}. Several different analyses have been
proposed for accomplishing these goals, and they all require some mechanism for
comparing brain graphs. 
 
Approaches to comparing brain graphs can be differentiated based on how they
treat the statistical relationships between edges. One such approach, referred
to as \emph{bag of edges}, is to treat each edge in the brain graph as a sample
from some random variable. Thus, a set of $N$ brain graphs each with $M$ edges
will have $N$ observations for each of the $M$ random variables. In this case,
the adjacency (or similarity) matrix that describes the brain graphs can be
flattened into a vector representation and any of the well explored similarity
or dissimilarity metrics can be applied to the data \cite{Ravindran}. One of
the benefits of this representation is the ability to treat each edge as
independent of all other edges and to compare graphs using mass univariate
analysis, in which, a separate univariate statistical test (e.g. t-test, anova,
or ancova) is performed at each edge. This will result in a very large number
of comparisons and an appropriate correction for multiple comparisons, such as
Network Based Statistic \cite{Zalesky2011}, Spatial Pairwise Clustering
\cite{Zalesky2012}, Statistical Parametric Networks \cite{Ginestet2013}, or
group-wise false discovery rate \cite{}, must be employed to control the number
of false positives. Alternatively the interdependencies between edges can be
modeled at the node level using multivariate distance multiple regression
(MDMR) \cite{Shehzad2014}, or across all edges using machine learning methods
\cite{Craddock2009, Dosenbach2010, Richiardi2011}.


 Despite the successful
application of this technique, a drawback of representing a brain graph as a
bag of edges is that this representation throws away all information about the
structure of the graph. Being able to
retain these graph structures within an analysis commonly known as Frequent
Subgraph Mining (FSM) has facilitated the discovery of features that better
discriminated between different groups of graphs \cite{Harrison2013}. For
instance, \cite{Bogdanov2014} were able to identify discriminative subgraphs
from functional connectivity graphs that had a high predictive power for high
versus low learners given specific motor tasks. \cite{Richiardi2013} outlines
other approaches that take the graph structure into account e.g. the graph edit
distance and a number of different graph kernels. All these methods are under
active development and have not been widely adapted by the connectomics

Another approach for graph similarity using all the vertexes involves computing
a set of \emph{graph-invariants} such as node centrality, modality, global
efficiency, etc. and using the values of these measures to represent the graph
\cite{rubinov}\cite{bullmoreReview}. Depending on the invariant used, this
approach may permit the direct comparison of graphs that are not aligned.
Another advantage is that invariants substantially reduce the dimensionality of
the graph comparison problem. On the other hand, representing the graph using
its computed invariants throws away information about that graph's vertex
labels \cite{Vogelstein2012}. Moreover, after computing these invariants it is
often unclear how they can be interpreted biologically. It is important that
the invariant used matches the relationships represented by the graph. Since,
edges in functional brain graphs represent statistical dependencies between
nodes and not anatomical connections, many of the path based invariants do not
make sense, as indirect relationships are not interpretable \cite{}. For
example, the relationships $A \leftrightarrow B$ and $B \leftrightarrow C$ do
not imply that there is a path between nodes $A$ and $C$, if a statistical
relationship between $A$ and $C$ were to exist they would be connected
directly.   

\subsubsection{Prediction}

Resting state fMRI and iFC analyses are most commonly applied to studying
clinical disorders and to this end, the ultimate goal is the identification of
biomarkers of disease state, severity, and prognosis\cite{DiMartino}. Prediction
 modelling has become a popular analysis method because it most
directly addresses the question of biomarker
efficacy\cite{craddock,Dosenbach,review}. Additionally, the prediction
framework provides a principled means for validating multivariate models that
more accurately deal with the statistical dependencies between edges than mass
univariate techniques, all while obviating the need to correct for multiple
comparisons. 

The general predictive framework involves learning a relationship between a
\emph{training} set of brain graphs and a corresponding categorical or
continuous variable. The features for the brain graphs can be (1) a set of
topological properties from each brain graph \cite{Cecci2009, Bassett2012}, (2)
a vector embedding of the brain graphs \cite{Richiadi2013,Luo2003, Craddock2009}, or (3) the
result of passing the brain graphs through a graph kernel \cite{}. The learnt
model is then applied to an independent \emph{testing} set of brain graphs to
decode or \emph{predict} their corresponding value of the variable. These
values are compared to their "true" values to estimate \emph{prediction
accuracy} - a measure of how well the model generalizes to know data. Several
different strategies can be employed to split the data into training and
testing datasets, although leave-one-out cross-validation has high variance and
should be avoided \cite{}. 

A variety of different machine learning algorithms have been applied to analyzing brain graphs in this manner,
but by far the most commonly employed has been support vector machines\cite{DiMartino}. Although these methods 
offer excellent prediction accuracy, they are often black boxes, for which the information that is used to make the
predictions is not easily discernable. The extraction of neuroscientifically
meaningful information from the learnt model cab be added by employing sparse methods and feature
selection to reduce the input variables to only those that are
essential for prediction.  There is still considerable work to be performed in
improving the extraction of information from these models, for developing
techniques that permit multiple labels to be considered jointly, and developing
kernels for measuring distances between graphs.

\subsubsubsection{specificity, better controls}

As a quick aside, it is important to keep in mind a few common analytical and experimental decisions that 
limit the utility of the putative biomarkers learned through predictive modeling. Generalization ability is
most commonly used to measure the quality of predictive models, but since this measure doesn't consider the
prevalance of the disorder in the population, it doesn't provide an accurate picture of how well a clinical 
diagnostic based on the model would perform. Instead it is important to estimate positive and negative predictive 
values \cite{Grimes, Altman} using disease prevalence information from resources such as Centers for Disease Control and 
Prevention Mortality and Morbidity Weekly Reports. Also, the majority of neuroimaging studies are designed to 
differentiate between an ultra-healthy cohort and a single severely-ill population, which further waters down
estimates of specificity. Instead it is also important to validate a biomarker's ability to differentiate between
several different disease populations - a very understudied area of connectomes research. Lastly, most predictive
modeling based explorations of the connectome are classifier based, which is very sensitive to noisy labels. Methods 
which incoporate some measure of label uncertainty or are robust to noisy labels are needed to help deal with this confound.


\subsubsubsection{dimensions} [ALT: A PSYCHIATRIC NOSOLOGY IN CRISIS: CHALLENGES AND POTENTIAL]
With the growing uncertainty about the biological validity of classical categorizations
of mental health disorders, there is a growing focus on symptoms that can be measured dimensionally. This 
Research Domain Criteria (RDoC) has become a major focus of the NIMH, and will no doubt engender a major shift in the 
manner in which connectomes experiments are performed. In the context of predicitive modeling this translates into change in
focus toward regression mdoels, which to date have been under utilizied in analyses of connectomes. But this dissatisfaction
with extant clinical categories, opens up a new broad opportunity for redefining clinical populations based on their 
biology rather than their symptomatology.

[CONSIDER COMBINING THE DIMENSIONS AND BLOBBING PIECES; THEY ARE BOTH SOLUTIONS TO THE CHALLENGE OF REDEFINING THE PSYCHIATRIC
NOMENCLATURE. FOLKS WHO PUSH ON DIMENSIONALITY ONLY SUFFER FROM THE FACT THAT THEY HAVE NO VISION THAT THE BIOLOGY AND/OR 
BETTER PHENOTYPING COMBINED WITH BIG DATA CAN BE USED TO REDRAW CATEGORIES].


Most prediction modeling in connectomes research has focused on classifier problems, with few studies using
regression frameworks. 

\subsubsection{blobbing}

A very under explored area of study is distinguishing between different popl

Predictive models are 
typically validated 


Several limitations of neuroimaging data, and the manner in which predicitive modeling analyses are 
commonly employed, limit the utility of the putative biomarkers that they learn. Probably the foremost
issue is that they are typically validated using cross-validation generalization accuracy, which does not
consider disease prevalence in their calculation, and thus is not informative about how well the classifier
would perform as a clinical diagnostic. For example, given a mental disorder with a high prevalence
(ADHD, 7.2\%) the probability of receving a false positive on a test with 100\% sensitivity and 90\% specificity
is .56, and for less common disorders (Autism, 1\%) the probability of a false positive becomes almost .91 \cite{grimes,altmanbland}.
Valuable information on prevalence of different disorders can be found from Centers for Disease Control and Prevention Mortality
and Morbidity Weekly Reports.  

Most neuroimaging studies of disease populations acquire data from an ultra-healthy cohort that is compared to
a severely-ill population. Although this strategy is ideal for maximizing power in inferential statistics, it 

\subsection{Turtles all the way down}

- Controversies for preprocessing
- Role of graph alignment
- Batch effects
- Dimensionality reduction
- Lack of a gold standard

With the growing prevalence of openly shared data, the desire to compare data
acquired on different scanners, with different scanning protocols, and at
different sites is becoming common. Although early studies have found that the
amount of biological variation between individuals outweighs between site
variation\cite{biswal}, there is still a need to standardize datasets against
batch effects\cite{YanStand}. Ideally, this will entail the acquisition of
calibration data, but until an accepted standard on what this calibration data
will include exists and its collection becomes commonplace, post-hoc
statistical corrections will be essential. Understanding the nature of batch
effects and developing adequate correction strategies is in its infancy and
there is substantial need for contributions in this area.  

No matter what type of method is being applied, iFC analyses are plagued by the
curse of dimensionality and an inability to validate their results with a gold
standard. Functional MRI datasets typically involve hundreds or thousands of
scans, each of which consist of a time series of hundreds of 3D brain volumes,
that each contain measures from hundreds of thousands of brain locations
(variables). Whether performing an analysis within a scan, or across scans, the
number of variables (N) is much larger than the number of observations (P).
Different analyses have approached the dimensionality problems in different
ways, but there is no consensus on the best algorithm for brain data. Indeed
there isn't a consensus about the best methods to use for any step of iFC
analyses, and this lack of agreement is due to a lack of a ground truth or gold
standard for comparing different techniques. Although several simulations have
been proposed, there is always some skepticism about their accuracy and
comprehensiveness. In their place, cross validation techniques have been
utilized to compare methods based on some measure of reproducibility and/or
generalizability.

A considerable drawback of using data-driven approaches for defining brain
regions, is that the number of regions in the clustering solution, or
conversely their size, must be specified. The choice of node size has been
shown to have a considerable effect on the resulting graph topology
\cite{Zalesky} and can also affect the outcome of an analysis\cite{Cecci_2009}.
There are a variety of mechanisms for optimizing the number of brain areas in a
parcellation, many of which using cross-validation methods, but in application
they do not tend to converge to a single optimum, and instead provide a range
of parcellations that are suitable\cite{craddock2012}. The choice of
parcellation used will strictly depend on the problem to be solved by that
analysis and the amount of error that one can allow when interpreting the
results. 


\section{MAKING DATA SHARING ACCESSIBLE TO DATA SCIENTISTS}

As with other big data problems, the analysis of the type of brain data
described above will benefit significantly from open science and open data. By
open science we mean the process of making intermediate and end data products
available to researchers outside the lab that initiated the investigation and
by open data we mean data sharing, data reuse, allowing groups other than the
ones that acquired the data to further analyze freely available data
\cite{Milham2012}. Both open science and open data have the benefit of
facilitating the creation of new questions to answer from a data set that the
people who acquired the data didn't conceive. A great example of this is the
bioinformatics community, which came into existence due to the availability of
large genomics datasets that were publicly available and could be used by
anybody who possessed the skills and knowledge to mine them\cite{VanHorn2013}.
This brought scientists and researchers from disparate fields (ranging from
biology, statistics, to computer science and engineering) together. The same
can be aspired for the brain imaging community by making data acquired in
different labs publicly available. At the moment there are a few public
databases such as the fMRI data center \footnote{fmri datacenter webaddress},
LONI IDA \footnote{LONI webaddress}, 1000 Functional Connectomes Project
(FCP)\footnote{\url{http://fcon_1000.projects.nitrc.org}}, the International
Neuroimaging Data-sharing Initiative (INDI)\footnote{INDI webaddress}, ABIDE
preprocessed\footnote{\url{http://preprocessed-connectomes-project.github.io/abide/}},
Human Connectome Project \footnote{HCP webaddress}, ADHD 200 preprocessed
\footnote{\url{http://neurobureau.projects.nitrc.org/ADHD200/Introduction.html}},CORR
\footnote{\url{http://fcon_1000.projects.nitrc.org/indi/CoRR/html/index.html}}.
Among these, there are a few such as the data from the Human Connectome Project
and the Cameron data set, which contain clean and tidy data. The latter will
speed up knowledge discovery and new method development by alleviating the need
for newcomers to acquire the knowledge and skills required to properly clean
raw MRI data. 

\section {Conclusion}
Functional connectomics is now a data science. As highlighted in this review, classic data science challenges related to data characterization, data reduction, and data classification 
are rapidly emerging as rate-limiting steps for this burgeoning field and its promises to transform the clinical realm. It is our hope that the recent augmentation of open science data-sharing initiatives with preprocessing efforts explicitly focussed on the removal of of common barriers to entry for data scientists (e.g., domain-specific knowledge, computational demands of image processing), will serve as an undeniable invitation and request for the involvement of the broader data science community.
 


